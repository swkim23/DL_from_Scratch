	\clearpage
	\section{퍼셉트론}
	
	\subsection{키워드}
	이번 장에서는 다음 키워드를 이해했다면, 그대는 성공~
	\begin{itemize}
		\item 노드(Node)
		\item 가중치(Weight)
		\item 편향, 임계값(Bias, $\theta$)
	\end{itemize}
	
	\subsection{정의}
	퍼셉트론은 다수의 입력을 받아 ``자체 연산''을 통해 출력 신호를 만들어 낸다. 이때 출력 신호는 0 또는 1이다. 입력 신호는 임의의 수만큼 받을 수 있다. 퍼셉트론은 각 입력 신호에 가중치를 곱한 값들을 합한다. 그리고 가중치를 곱한 값의 합이 임계값 보다 크면 1 임계값보다 작으면 0을 출력 신호로 만든다. 여기에서 입력 신호의 가중치를 곱하고 그들을 더하여 임계값과 비교하는 것을 퍼셉트론의 ``자체 연산''이다.
	


	편향(bias)는 뉴런이 얼마나 쉽게 활성화 되는지를 제어한다.
	가중치는 각 신호의 영향력을 의미한다.
	
	
	Perceptron은 복잡한 함수도 표현이 가능하다. 심지어 다층 퍼셉트론은 컴퓨터도 표현 가능하다.
	하지만, 퍼셉트론의 가중치를 설정하는 것은 여전히 사람에 의해서만 가능하다.
	
	신경망은 데이터로 부터 퍼셉트론의 가중치 설정을 자동으로 한다.
	